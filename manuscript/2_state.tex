\section{State-of-the-art and progress beyond}\label{stateoftheart}
This section provides an overview of relevant scientific literature for this paper's scope. It emphasizes two essential subjects, namely, the role of natural and green gases in sustainable energy systems (Section \ref{state:1}) and the modeling of gas networks from a system perspective point of view (Section \ref{state:2}). The novelties of this work and progress beyond state-of-the-art are also presented (Section \ref{state:3}). 

\subsection{Natural and green gases in sustainable energy systems}\label{state:1}
It is debatable whether natural gas will play a significant role in the energy transition over the next few decades, and if so, under what conditions. Gürsan and Gooyert \cite{gursan2021systemic} provide a recent and concise review of the state-of-the-art in the role of natural gas in reducing CO\textsubscript{2} emissions from energy systems. Kotek et al. \cite{kotek2019european} conduct a study on the European natural gas infrastructure in the context of energy transition. Already in 2012, Stephenson et al. \cite{stephenson2012greenwashing} discuss natural gas as a transition fuel in the sustainable transformation of energy systems. They concluded that a natural gas climate solution is unsubstantiated. This is also reflected in a large number of studies on cost-optimal energy supply until 2050. Auer et al. \cite{auer2020development}, for example, investigate the European energy supply until 2050 for various decarbonization scenarios under the remaining European fraction of the CO\textsubscript{2} budget and discover that natural gas is almost completely replaced in the primary energy demand in 2040.\vspace{0.35cm}

Green gases are becoming increasingly important, as evidenced by not only the results of Auer et al. for Europe but also, those of Zhang et al. \cite{zhang2022potential} for China. Against this background, it is certainly possible to see existing natural gas networks as a crucial part of the energy transition to transport and deliver green gases. Recently, Quintino et al. \cite{quintino2021aspects} elaborate on aspects of green gas introduction in natural gas networks. Dodds and McDowall \cite{dodds2013future} examine the long-term future of gas networks and state that the most cost-effective strategy might be to convert the networks to deliver green gases. Interestingly, Dodds and McDowall find in their scenarios that hydrogen injection into gas networks has only a small role and low impact on gas networks. \added[]{Kolb et al. \cite{kolb2021scenarios} examine the integration of green gases into a natural gas market. They provide not only a comprehensive literature review of the future role of renewable gases but also show that the carbon price is one of the most promising ways to promote renewable gases.} \deleted{Similarly,} Mac Kinnon et al. \cite{mac2018role} investigate the role of natural gas networks in mitigating greenhouse gas emissions. Gillessen et al. \cite{gillessen2019natural} elaborate on the role of natural gas as a bridge to sustainable energy systems and related infrastructure expansion of gas networks. Gondal \cite{gondal2019hydrogen} studies the hydrogen integration into gas transmission networks.\footnote{Particularly, Gondal states that (i) at the transmission network level, compressors are the determinant element and limit the value of hydrogen by 10\%; (ii) at the distribution network level, pipelines and storage elements allow shares up to 50\% of hydrogen; and (iii) at the level of end use appliances, a tolerant range and share of 20-50\% of hydrogen is possible.}\vspace{0.35cm}

Nonetheless, although the expected potential of green gases exists, it is nowhere large enough to replace the current amount of natural gas in the energy supply. Accordingly, the discussion of existing natural gas networks may and should include decommissioning as part of the solution space. Furthermore, this possibility should no longer be seen as a taboo subject but rather as a real decision option that can even be argued from a techno-economic point of view. Giehl et al. \cite{giehl2021modelling} examine cost-optimal gas networks and focus particularly on the distribution network level, finding a declining need for gas distribution networks in their future scenarios. Zwickl-Bernhard and Auer \cite{zwickl2022demystifying} present the decommissioning of a gas distribution network in an urban area at the community level. Feijoo et al. \cite{feijoo2018future} find risks of underutilization of gas networks (i.e., pipeline capacities) in a low-carbon future economy even at the interstate and transmission level. Brosig et al. \cite{brosig2017benchmark} compares the cost-effectiveness of different future pathways between expansion and decommissioning of the gas grid network.\vspace{0.3cm}

In this context, local renewable energy sources and technologies are becoming increasingly important. For example, district heating contributes in densely populated and urban areas to the decrease of natural gas in the supply of energy service needs. Möller and Lund \cite{moller2010conversion} examine the conversion of individual natural gas heating units to district heating. Hofmann et al. \cite{hofmann2014potential} show the use of geothermal sources for heat generation for both residential and industrial. At the national level, Geyer et al. \cite{geyer2021100} presents scenarios, energy carriers and infrastructure requirements for a completely renewable energy-based industry sector. Rahnama et al. \cite{rahnama2021pulp} show particularly the reduction of gas demands and associated CO\textsubscript{2} emission for the pulp and paper industry by electrification of energy service needs. Bachner \cite{bachner2020risk} et al. focus on the replacement of gas and other fossil fuels in the steel and electricity sector from a macroeconomic perspective\footnote{In the context of a decarbonized electricity supply, Qadrdan et al. \cite{qadrdan2015impact} investigate the impact of transitioning to a low-carbon electricity sector on gas network infrastructure. Particularly, the authors focus on the gas network in Great Britain and find that despite the declining gas demand, the peak gas demand remains unchanged.}.\vspace{0.35cm}

Findings of the literature in the previous paragraph indicate that large portions of natural gas demands can, in principle, be substituted by sustainable alternatives. Against this background and considering that natural gas networks are regulated entities of the energy system, are capital intensive, and therefore require long-term strategies or planning, avoidance of stop-and-go policy is crucial. Exemplarily, Then et al. \cite{then2020impact} study the operator strategy and economic viability of gas networks in face of decreasing gas demands. Hickey et al. \cite{hickey2019there} identify significant challenges and risks to policymakers and investors in using gas networks in sustainable energy systems encompassing the risk of stranded assets resulting not only from declining gas demand but also from changes in regulation and how tariffs are allocated. Hausfather \cite{hausfather2015bounding} focuses on the policy decisions for natural gas and its network infrastructure until 2030 as they irreversibly impact the future of natural and synthetic gas in the period 2030 to 2050\footnote{Moreover, Hausfather concludes that policy decisions are needed leading to decarbonization of natural gas no later than 2030.}. Glachant et al. \cite{glachant2014gas} study and identify the fundamental reasons for diverging gas network and market developments. Mos{\'a}cula et al. \cite{mosacula2018designing} propose a novel methodology for gas network charges design, which builds on economic efficiency as the main principle.\vspace{0.3cm}

\added[]{A few selected works on energy and network security of natural gas are mentioned here without claiming completeness. Hutagalung et al. \cite{hutagalung2017economic} deal with investments in natural gas infrastructure and their implications on energy security and economic growth.} Tata and DeCotis \cite{tata2019natural} focus on risks and responsibilities associated with natural gas infrastructure development. \added[]{They do not advocate for or against using natural gas but argue that all energy supply options have risks that must be considered when deciding which fuel to use.} Sacco et al. \cite{sacco2019portfolio} analyze the risks associated with maintenance gas networks. Sesini et al. \cite{sesini2020impact} assess resilience and security in gas network systems. \replaced[]{They examine the impact of a real demand shock on the European natural gas network, focusing on liquefied natural gas and storage.}{The key findings can be summarized as decisions on natural gas infrastructure development should not be made through a single-lens view.}

\subsection{Modeling gas networks}\label{state:2}
Particularly, the previous paragraph regarding the challenges and risks of long-term planning of gas networks provides the starting point for this section dedicated to modeling and simulation of gas networks. R{\'\i}os-Mercado \cite{rios2015optimization} present a comprehensive state-of-the-art review on the optimization of natural gas networks encompassing both the transmission and distribution network level. Osiadacz and Gorecki \cite{osiadacz1995optimization} provide an even broader summary of gas network optimization modeling approaches. Particularly, they mention heuristic, continuous and discrete methods of the optimal design of gas networks. Feijoo et al. \cite{feijoo2016north} propose a long-term partial equilibrium model that allows for endogenous
gas network infrastructure expansion and nonlinear cost functions. Fügenschuh et al. \cite{fugenschuh2011gas} develop an optimization model with quadratic formulation. Fodstad et al. \cite{fodstad2016stochastic} and Aßmann et al. \cite{assmann2019decomposable} use stochastic optimization including gas demand uncertainties in the optimization of gas networks. Latter use a decomposable robust two-stage optimization model. Von Wald et al. \cite{von2022optimal} propose a multiperiod planning framework for decarbonization of integrated gas and electric energy systems. \added[]{Similarly, the work of Cavana et al. \cite{cavana2021electrical} is aimed at sector coupling. In their work, the authors emphasize the ability of the gas network for hydrogen blending to enable greater integration of renewable energy.}\vspace{0.35cm}

The long-term planning of gas networks is exemplarily shown by Hubner and Haubrich \cite{hubner2008long} and Giehl et al. \cite{giehl2021modelling}. \deleted[]{The latter proposes a greenfield approach and optimization model for gas networks without considering the existing network infrastructure (i.e., from scratch).} Mikolajkov{\'a} et al. \cite{mikolajkova2017optimization} show the optimization of a natural gas distribution network with the potential future extension of the transmission network level. Kashani et al. \cite{kashani2014techno} present the techno-economical and environmental optimization of natural gas network operation. Farsi et al. \cite{farsi2007cost} show a national case study regarding the cost efficiency of gas distribution networks. Particularly, they emphasize the impact of customer density and network size in the Swiss gas distribution sector. Odetayo et al. \cite{odetayo2018real} show the modeling flexibilities of gas networks for energy system operation. Di{\'e}guez et al. \cite{dieguez2021modelling} show the modeling of decarbonization transition in a national integrated energy system including hourly operational resolution of gas networks. Yusta and Beyza \cite{yusta2021optimal} emphasize the modeling of large-scale gas storage facilities by a dynamic approach. Kerdan et al. \cite{kerdan2019modelling} link a spatially resolved gas infrastructure optimization model with an energy system model.\vspace{0.3cm}

\added[]{This paragraph concludes the literature review with a discussion of existing approaches and models used to achieve similar objectives to this study. It becomes clear that there are relevant studies in this area but also that there is currently a lack of work in the literature explicitly addressing decommissioning and reinvestment decisions for gas networks. For example, Giehl et al. \cite{giehl2021modelling} provide an in-depth analysis of the impact of energy system decarbonization on gas distribution networks. For a German case study, they show a decreasing need for gas distribution networks until 2050. A strength of their approach is certainly the comprehensive model they use, called "DINO". However, they use a greenfield approach and do not consider existing networks and pipelines. Then et al. \cite{then2020impact} examine the impact of different gas network operator strategies on gas network profitability in the face of declining demand. They carry out comprehensive gas network planning under different strategies but focus on network charges in terms of strategy. They also use commercial software and estimate a functional relationship between required network length and demand. Bouacida et al. \cite{bouacida2022impacts} investigate the impact of decarbonization on gas networks and costs. Based on a review of French and German decarbonization scenarios, they examine the impact on gas networks and estimate the change in gas prices for end users. They also argue that the latter will increase, accelerating the substitution of natural gas. In their paper, Oberle et al. \cite{oberle2022can} ask whether industrial gas demand can keep gas distribution networks alive or not? They focus on the development of gas demand in Germany until 2050. However, their analysis focuses more on the implications for gas networks than detailed network modeling. Given this, there is no existing work proposing a method that could, to the best of our knowledge, be used to address the objectives of this work.}

\subsection{Progress beyond the state-of-the-art}\label{state:3}
Based on the literature review, the novelties of this work can be summed up as follows:
\begin{itemize}
	\item A cost-effective trajectory of existing gas network infrastructure is modeled considering the expectation of both declining gas demands resulting from the defossilization of energy services and the increasing but limited integration of green gases, such as synthetic gas and hydrogen.  
	\item Since existing gas network infrastructure requires the refurbishment of its gas pipelines due to the expiration of the technical lifetime, it is shown how the gas network operator decides from a techno-economic point of view between decommissioning and refurbishment investment of gas pipelines at different network and pressure levels. Especially, the transmission, high-pressure, and mid-pressure network levels are analyzed. \added[]{The assumed pressure levels are between} \SI{64}{bar} \added[]{for the transmission and high-pressure network levels, and} \SI{4}{bar} \added[]{for the mid-pressure network level.}
	\item The optimization of a cost-effective trajectory of existing gas network infrastructure includes, the gas network operator's decision between supplying or not supplying available gas demand (i.e., disconnection from the gas network by decommissioning gas pipelines and implicitly implementing stand-alone gas supply alternatives). Particularly, the long-term planning horizon of the model allows for investigating this trade-off decision between investment/capital costs, related book values, and expected revenue and purchase streams for individual gas pipelines.
	\item The application of the proposed model on a real test bed in a NUTS2 region in Austria until 2050 provides useful insights that can be used directly by decision and policymakers. The investigated test bed is representative of other gas networks since it comprises, on the one hand, gas demands that are supplied in different end-user sectors, and, on the other hand, encompasses different gas network/pressure levels. Accordingly, the transmission network level is also considered as the test bed gas network is linked to neighboring countries.  
\end{itemize}

% falschen erwartungshaltung bezüglich des umwidmungspotenzial von methanleitungen für den transport von wasserstoff weil derzeit sehr viel auf transmission network level passiert
