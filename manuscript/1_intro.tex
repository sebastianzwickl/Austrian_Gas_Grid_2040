\section{Introduction}

Natural gas grids in Europe have been operated economically in recent decades. There are two main reasons for this. First, natural gas has been an extremely cheap energy source, mainly due to its unlimited availability through imports. As a result, natural gas has been demanded in large quantities to cover different energy services. Second, the transport of natural gas through pipelines over short and long distances to cover demand was and is technically efficient and economically cheap. However, the future of natural gas grids and their economic efficiency is unclear today. The main argument for smaller gas grids in the future is the expected decline in demand for natural gas as Europe aims to meet its ambitious decarbonization targets. 


das ist sicher ein genereller trend und trasferabel do all the european countries
in einigen ländern, wie deutschland, italien und auch österreich, wird die renewable gas production eine rolle spielen.
diese ist inbesondere in der fläche zu erwarten, könnte zu einem renewable betrieb von existierenden gasmetzen führen.
gleichzeitig würde das zu einem netz in der fläche führen 
außerdem spielt natürlich das thema hydrogen noch eine wichtige rolle. dessen rolle auf die gasnetze muss vermutlich unterschiedlich auf den netzebene gesehenm werden.
auf der fernleitungsebene zeichnet sich ein sehr klares bild ab
umwidmung da man stränge hat 
hydrogen backbone etc.
auf den untergelagerten netzebene (high. mid, und low pressure) ist die entwicklung unklar 
nicht nur wo die verbräuche sitzen weiterbetrieb von leitungen, sondern auch durch die einspeisung

the goal of this paper is...

the main research questions are:

the method applied to answer

numerical example

organizer of the paper

% 208 Anlagen Biomethan, mit einer eingespeisten Energiemenge von 9.591 GWh/a [7]. aus dem IEWT Paper kopiert

% Die Einspeisung von Biomethan in das Gasnetz ist in Ländern wie Deutschland, Schweden und Österreich bereits gängige Praxis [17].

% Die Mindestgröße von Biogasaufbereitungsanlagen (250 Nm 3 /h) für eine wirtschaftliche Aufbereitung von Biogas zu Biomethan [21] übersteigt in der Regel die Größe der Biogasproduktionskapazität einzelner Biogasfermenter, die in Deutschland durchschnittlich 180 Nm 3 /h beträgt [22].

% Different components of a gas system, such as underground and over ground storage, gas turbines and engines, domestic and indus trialappliances, compressors, and valves, are usually designed to transport and operate on natural gas with a consistent quality. Therefore, at present it is not possible to specify limiting values for alternative gas injections which would be valid for all parts of the gas infrastructure.





%Adherence to the remaining CO\textsubscript{2} budget of the Paris Agreement requires rapid defossilization of the energy system \cite{rockstrom2017roadmap}. In Europe, the \textit{Fit for 55} package \cite{european_commission_european_2019} and the \textit{EU Green Deal} \cite{greendeal} define the mid- and long-term goals for a transition to a sustainable energy supply until the middle of the century. These goals include a reduction of CO\textsubscript{2} emissions by 55\% compared to those in 1990 by 2030 and climate neutrality in 2050. In light of this, the question of the concrete design of measures to achieve these goals arises \cite{hainsch2022energy}. Numerous scientific works have already been dedicated to the analysis of sustainable alternatives for the provision of energy service needs, which currently rely on fossil fuels. Abas et al. \cite{abas2015review} provide a comprehensive review of fossil fuels as a primary energy source in the energy supply chain and future energy technologies. Corresponding studies, \added[]{(e.g., \cite{hainsch2022energy} and \cite{abas2015review}),} \replaced[]{often}{frequently} focus primarily on the renewable energy technology portfolio that provides energy service needs in the future. We essentially use these as the starting point of our analysis here, investigating implications of expected declining coverage of energy services by natural gas, a fossil fuel, on its transmission and distribution network infrastructure.\vspace{0.35cm}

%Natural gas is undeniably one of the pillars of existing energy systems, but it is being fundamentally challenged by the already established and ongoing decarbonization of energy systems. Furthermore, as part of the sustainable transition, natural gas and its role are expected to undergo significant transformation.\footnote{McGlade and Ekins \cite{mcglade2015geographical} state that half of natural gas reserves should remain unused from 2010 to 2050 in order to meet at least the less ambitious 2.0°C climate target from the Paris Agreement.} However, presently, it is not clear which exact trajectory natural gas will take until 2050 \cite{safari2019natural}.\footnote{Exemplarily, Kumar et al. \cite{kumar2011current} see natural gas as an important bridging fuel to a sustainable energy system, in some cases even after 2050. By contrast, Stephenson et al. \cite{stephenson2012greenwashing} propose to abandon the transition fuel characterization of natural gas. D{\'\i}az et al. \cite{diaz2017we} follow this point of view since they find for the electricity sector that natural gas delivers little to no cost savings as a bridging fuel in a system that switches to wind and solar.} Two focal reasons/subjects are (i) various energy sectors currently use natural gas in the provision of energy services (e.g., generation of process heat or as a base material for industrial consumers, centralized generation of electricity and district heating, and decentralized supply of space heating and hot water demands), and it is not clear if and when exactly sustainable alternatives will be economically available, implemented, and realized in sufficient quantities \cite{mohseni2013competitiveness, gorre2019production}. (ii) Synthetic gases (including hydrogen) are seen as a promising alternative or supplement to natural gas usage, as they could be fed into existing transmission and distribution gas network infrastructure, although there are valid uncertainties regarding its amount of technical as well as economic potentials \cite{lux2020supply, blanco2018potential}. Because of that, the question is not only which energy sectors and energy services remain to use the limited quantities of natural (following the trend of defossilization) and synthetic (because of limited potentials gas) but also what gas network infrastructure will continue to be needed for their transport and distribution.\vspace{0.35cm}

%The goal of this paper is to contribute to scientific research on its future development and trajectory of gas network infrastructure with the expectation of decreasing natural gas demands and increasing the integration of green gases, such as synthetic gas and hydrogen. The emphasis is on gas network infrastructure, which ensures that various energy service needs (e.g., residential building heat, and industrial process heat) are met. This raises the question of which gas network infrastructure is required to meet non-substitutable natural gas demands when considering possible stand-alone natural gas supply options (e.g., delivery of liquefied natural gas by truck) \added{in order to avoid significant economic loss and stranded assets in the future. To meet demand, the network can transport either imported gas volumes or locally produced green gases such as biomethane.} Nonetheless, even if stand-alone solutions are viable alternatives in some cases, arbitrariness regarding the trajectory decisions of gas networks must be avoided, as they are not only assigned to critical infrastructure but also regulated and subject to long-term energy planning (lock-in effect).\vspace{0.35cm} 

%The primary objective of this work is to investigate the cost-effective trajectory of gas network infrastructure from a systemic viewpoint under a long-term planning horizon. Given necessary refurbishment investments in existing gas network infrastructure and pipelines due to their technical lifetimes, the main research question is which decommissioning and refurbishment investment decisions result in a cost-effective gas network infrastructure by 2050. Equally important in the analysis is the network operator's trade-off decision regarding whether available gas demands within the network area are supplied or not, as decommissioning of existing gas pipelines can be cost-effective but results in unsupplied gas demands. Consequently, three different model runs are performed, allowing a thorough comparison of various handling options in terms of gas demands not met by the network infrastructure.\footnote{Grid operator usually is a regulated entity. Thus, finally, it is a regulatory question/decision which cannot be taken by the grid operator alone (regulator).} Accordingly, analysis relies heavily on the shadow prices of gas supply at the local level within the network's nodes. \added{At present, existing studies on gas network infrastructure are insufficient, and the corresponding literature has not yet conducted a comprehensive comparison of different sub-policy implications for handling future gas demands by a quantitative analysis.}\vspace{0.35cm}

%The method used is the development of a linear optimization model. Thereby, the objective function is to minimize the network operator's net present value over time. Particularly, the optimal solution of the model includes the decommissioning and refurbishment investment decision of parts of the network and single pipelines. This includes deciding whether or not to supply available gas demand. The dual variables of the local gas balance constraints allow us to assess the techno-economic range of supply alternatives for each node in the network.\vspace{0.35cm}

%The numerical example examined is a small portion of the existing Austrian gas network infrastructure in the NUTS2 region Vorarlberg, Austria. This area is distinguished by a wide range of energy service requirements that are met by natural gas (e.g., residential, and industry). Furthermore, the gas network infrastructure includes not only high- and mid-pressure network connections but also cross-border connections to neighboring countries Germany and Lichtenstein (i.e., transmission network level). There is also the possibility of producing green gas and injecting it into the existing gas network infrastructure.\vspace{0.35cm}

%The paper is organized as follows. Section \ref{stateoftheart} provides an overview of the current state-of-the-art in scientific literature and outlines the novelties of this work beyond existing research. Section \ref{methodology} presents the materials and methods developed in this work, including, the model's mathematical formulation and description of different model runs. Section \ref{results} presents the results of this work encompassing different handlings of gas demands within the network. Section \ref{conclusions} synthesizes and discusses the results, concludes the work, and gives an outlook for future research.  