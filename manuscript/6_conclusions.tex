\section{Conclusions}\label{conclusions}
%The ongoing defossilization of the provision of energy services leads to declining natural gas demands. That and the expectation of very limited economically useable potentials of green gases, a transparent and critical discussion regarding the future development of existing gas networks without any taboos is needed. This work investigates the trajectory of gas networks in a test bed until 2050, including the decision of decommissioning parts of the existing gas networks. Particularly, the analysis is conducted out from the network operator's perspective and shows different network decommissioning or refurbishment options under the decision of supplying or not supplying available gas demands.\vspace{0.3cm}
%
%We find that smaller gas networks (in terms of pipeline capacity and network length) are needed in the future regardless of ensured supply. However, the results indicate a wide range of possible network developments until 2050 resulting from the handling of gas demand. This reveals crucial trade-off decisions for gas network operators in the future and includes, the decommissioning decision of gas pipelines despite possible gas demand. Moreover, the conducted analysis of shadow prices of the local gas balance constraint shows that a balance/trade-off between the cost-optimal gas network design with and without ensured supply could lead to a robust and economically competitive future of gas networks.\added[]{Ultimately, however, the size of the gas networks is determined not only by the demand side but also by the feed-in side and, thus, in the future, by the quantities in which green gases are economically available.}\vspace{0.3cm}
%
%Nevertheless, the results demonstrate that it is necessary to socialize network operators' costs under the remaining consumers connected to the network in the future. This fact has several important implications. First and foremost, that brings an additional cost component to consumers \added{(particularly relevant for the hard-to-abate energy sectors)}, which needs to be considered when dealing with the profitability of sustainable alternatives substituting natural gas. Analyses elaborating on trade-offs between natural gas and other sustainable supply options are often neglecting this network-related cost component, which brings a bias into the decision process.\vspace{0.3cm}
%
%\added{However, this research has several limitations and further improvements could be done regarding different aspects. In particular, f}uture work should investigate the development of gas demand in different sectors (e.g., building heat and industry) in more detail, bringing further insights into gas networks requirements and topologies until 2050. Additionally, the role of green gases could be enhanced. In particular, further work should include different types of gas pipelines associated not only with the network/pressure level but also with the quality of the gas transported.