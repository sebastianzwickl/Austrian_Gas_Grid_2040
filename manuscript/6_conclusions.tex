\section{Conclusions}\label{conclusions}
was es wert diese analyse durchzuführen: nicht nur mengen, sondern auch einspeisung und deren verortung.


die zukunft von erdgasnetzen bleibt eine der spannendsten fragen die sich durch die umsetzung der dekarbonisierung ergibt. 

unbestritten, wird es zu einer verkleinerung der erdgasnetze kommen.

auf der fernleitungsebene sehr eindeutig, dass eine umwidmung zu wasserstoff möglich ist, weil kapazitäten vorhanden sind. parallelstränge erlauben es

auf der verteilnetzebene nicht mehr so eindeutig. 

doppelstrukturen herauslösen weil netz oft redundanzen hat.

setzt man auf biomethane große netze weiter gebraucht.

dabei kommt es weniger auf die absoluten mengen an, sondern die verteilte einspeisung ist eher eine ja/nein entscheidung

teurere netze, selbst im elektrifizierung noch ein großes netz 

schaffen regional/lokal biomethan, genau abgestimmt wo weiterhin verbrauch bleibt

zukünftige arbeiten, diese regionalen cluster zu identifizieren
weitere technische details berücksichtigen, wie die druckentwicklung in schwächer ausgelasteten netzen, energie die gebraucht wird um druckhertzstellen, etc.
